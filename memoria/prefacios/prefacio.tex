\chapter*{}
%\thispagestyle{empty}
%\cleardoublepage

%\thispagestyle{empty}

%\begin{titlepage}
 
\newlength{\centeroffset}
\setlength{\centeroffset}{-0.5\oddsidemargin}
\addtolength{\centeroffset}{0.5\evensidemargin}
\thispagestyle{empty}

\noindent\hspace*{\centeroffset}\begin{minipage}{\textwidth}

\centering
\includegraphics[width=0.9\textwidth]{imagenes/logo_uoc.png}\\[1.4cm]

\textsc{ \Large TRABAJO FIN DE MÁSTER\\[0.2cm]}
\textsc{ INGENIERÍA INFORMÁTICA}\\[1cm]
% Upper part of the page
% 
% Title
{\Huge\bfseries Monitorización del consumo eléctrico en tiempo real\\
}
\noindent\rule[-1ex]{\textwidth}{3pt}\\[3.5ex]
{\large\bfseries Micropython, Esp32 y MQTT}
\end{minipage}

\vspace{2.5cm}
\noindent\hspace*{\centeroffset}\begin{minipage}{\textwidth}
\centering

\textbf{Autor}\\ {Adrián Bonel Bolívar}\\[2.5ex]
\textbf{Director}\\{Alvaro Gomez Pau}\\[2cm]
%\includegraphics[width=0.3\textwidth]{imagenes/etsiit_logo.png}\\[0.1cm]
\textsc{Sistemas encastados}\\
\textsc{---}\\
Barcelona, Enero de 2023
\end{minipage}
%\addtolength{\textwidth}{\centeroffset}
%\vspace{\stretch{2}}
\end{titlepage}






\cleardoublepage
\thispagestyle{empty}

\begin{center}
{\large\bfseries Monitorización del consumo eléctrico en tiempo real}\\
\end{center}
\begin{center}
Adrián Bonel Bolívar\\
\end{center}

%\vspace{0.7cm}
\noindent{\textbf{Palabras clave}: Micropython, Software embebido, Esp32, Acs712, MQTT, Flask, Python, App Web}\\

\vspace{0.7cm}
\noindent{\textbf{Resumen}}\\

A día de hoy, el coste de la electricidad ha aumentado a unos niveles que nadie hubiera imaginado hace tan solo unos años. Muchas familias cada vez más, necesitan ir con mucho cuidado sobre cuándo poner ciertos electrodomésticos en casa, para elegir la hora donde el precio está más bajo y así poder ahorrar en la factura de la luz. Saber cuánto consume cada electrodoméstico y como se refleja en el coste de la factura puede llegar a ser de gran utilidad. \\

Aquí es donde podemos situar el trabajo de este proyecto. Aunque a día de hoy existen sistemas inteligentes con los que gestionar y monitorizar la energía que se consume en el hogar, normalmente estos dispositivos son independientes unos de otros y no encontramos muchos que recopilen los datos. El trabajo de este proyecto, no solo se va a centrar en medir el consumo eléctrico de los dispositivos individualmente, si no que el objetivo será que toda esa información se pueda consultar desde una misma aplicación. \\

Planteamos por tanto el despliegue de sistemas embebidos capaces de medir el consumo eléctrico y que a su vez sean capaces de comunicarse con un servidor central donde recopilar todos los datos. El despliegue de estos sistemas se hará en un entorno real, donde se probará el funcionamiento de los dispositivos y se analizará el rendimiento de los mismos. \\

\cleardoublepage


\thispagestyle{empty}


\begin{center}
{\large\bfseries Real time measurement of power consumption}\\
\end{center}
\begin{center}
Adrián Bonel Bolívar\\
\end{center}

%\vspace{0.7cm}
\noindent{\textbf{Keywords}: Micropython, Embedded software, Esp32, Acs712, MQTT, Flask, Python, Web app}\\

\vspace{0.7cm}
\noindent{\textbf{Abstract}}\\

Today, the cost of electricity has risen to levels that no one would have imagined just a few years ago. Increasingly, many families need to be very careful about when to put certain devices in the house, to choose the time when the price is lower and thus be able to save on the electricity bill. Knowing how much each device consumes and how it is reflected in the cost of the bill can be very useful. \\

This is where we can place the work of this project. Although today there are intelligent systems with which to manage and monitor the energy consumed in the home, these devices are usually independent of each other and we do not find many that collect the data. The work of this project will not only focus on measuring the power consumption of individual devices, but the goal will be that all this information can be consulted from a single application. \\

We therefore propose the deployment of embedded systems capable of measuring electricity consumption and at the same time be able to communicate with a central server to collect all the data. The deployment of these systems will be done in a real environment, where the operation of the embedded systems will be tested and their performance will be analyzed. \\

\chapter*{}
\thispagestyle{empty}

\noindent\rule[-1ex]{\textwidth}{2pt}\\[4.5ex]

Yo, \textbf{Adrián Bonel Bolívar}, alumno del Máster en Ingeniería Informática de la \textbf{Universitat Oberta de Catalunya}, con DNI 76668939A, autorizo la
ubicación de la siguiente copia de mi Trabajo Fin de Máster en la biblioteca del centro para que pueda ser
consultada por las personas que lo deseen.

\vspace{6cm}

\noindent Fdo: Adrián Bonel Bolívar

\vspace{2cm}

\begin{flushright}
Barcelona a 30 de Noviembre de 2022.
\end{flushright}


%\chapter*{Agradecimientos}
%\thispagestyle{empty}

%       \vspace{1cm}


%Inventar cosas aqui



