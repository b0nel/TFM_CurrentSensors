\begin{titlepage}
\chapter{Análisis de costes}
Para el análisis de los costes, me he basado en la planificación inicial (solo tendré en cuenta las horas de realización del proyecto y no las horas de realización de este documento).\\

1. En primer lugar tenemos que considerar que el trabajo de Fin de Máster cuenta como 12 ECTS, lo que debería equivaler a 300 horas de trabajo (25h por crédito). \\

2. Inicialmente se estimo un total de 43 dias (344 horas) para la realización del proyecto. Aquí ya podemos observar que no se realizó una buena planificación en cuanto a las horas disponibles para la realización del proyecto. Las 44h de más que se plantearon inicialmente tambien las tendré en cuenta.\\

3. El coste de trabajo de cada hora, para un ingeniero informático con experiencia y en base a la experiencia que tengo en el mercado laboral, consideraré un coste por hora de 20 euros brutos, lo que supone un coste salarial de 20\euro /h x 344h = 6880\euro.\\

4. Si considerasemos este proyecto para su realización como algo profesional, deberiamos entonces considerar algunos costes indirectos relacionados con el desarrollo de la actividad, tales como agua, luz, conexión a internet, alquiler de equipos, etc. A grosso modo podriamos estimar unos 1200 euros.\\

5. Para la realización de este proyecto, el uso del equipo informatico para el desarrollo e investigación del proyecto tambien se ha de tener en cuenta. Para cada uno de ellos se calcula el coste estableciendo su vida útil y el tiempo de uso como variables que lo componen. Se supone, para todos, una vida útil de 36 meses y un tiempo de uso de 6 meses. La proporción a seguir es (Meses uso / Vida útil) * Coste de Adquisición.\\

Ordenador de sobremesa: 
\begin{itemize}
	\item Precio Adquisición = 2.000 \euro
	\item Coste = 333 \euro
\end{itemize}

Raspberry: 
\begin{itemize}
	\item Precio Adquisición = 40 \euro
	\item Coste= 7 \euro 
\end{itemize}

Ubuntu: 
\begin{itemize}
	\item Precio Adquisición = 0 \euro
	\item Coste= 0 \euro  
\end{itemize}

Raspbian OS: 
\begin{itemize}
	\item Precio Adquisición = 0 \euro
	\item Coste= 0 \euro  
\end{itemize}

Fuente de alimentación: 
\begin{itemize}
	\item Precio Adquisición = 150 \euro
	\item Coste= 25 \euro  
\end{itemize}

Subtotal: 365 \euro

6. Para los componentes electronicos, consideramos su precio de adquisición como coste total.\\

Sensores ACS712 5A:
\begin{itemize}
	\item Coste=  13\euro  
\end{itemize}

Sensores ACS712 30A:
\begin{itemize}
	\item Coste=  13\euro  
\end{itemize}

Sensores ZMPT101B:
\begin{itemize}
	\item Coste=  24\euro  
\end{itemize}

Microcontroladores ESP32:
\begin{itemize}
	\item Coste=  28\euro  
\end{itemize}

Caja de resistencias:
\begin{itemize}
	\item Coste=  15\euro  
\end{itemize}

Caja de condensadores:
\begin{itemize}
	\item Coste=  10\euro  
\end{itemize}

Reles:
\begin{itemize}
	\item Coste=  12\euro  
\end{itemize}

Subtotal: 115 \euro \\

Sumando todo, el coste total del desarrollo del proyecto se podria estimar en: \\

Salario: 6880\euro

Costes indirectos: 1200\euro

Coste de equipos: 365\euro

Coste de hardware electronico: 115\euro

TOTAL: 8560 \euro


\end{titlepage}